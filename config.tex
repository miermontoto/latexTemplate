\pgfplotsset{compat=1.18}
\usepgfplotslibrary{dateplot}

% colores
\definecolor{gris}{RGB}{220,220,220}
\definecolor{backcolour}{rgb}{0.95,0.95,0.92}
\definecolor{darkgreen}{rgb}{0.0, 0.5, 0.0}

\setcounter{secnumdepth}{3} % Para permitir numerar las sub-subsecciones

% Modifica el nombre de los índices al castellano
\addto\captionsspanish{
  \renewcommand{\contentsname}{Índice de contenido}
  \renewcommand{\listfigurename}{Índice de figuras}
  \renewcommand{\listtablename}{Índice de tablas}
  \renewcommand{\lstlistingname}{Listado}
  \renewcommand{\lstlistlistingname}{Índice de listados}
}

% Formateo de los nombres de los apartados:
\titleformat{\chapter}[block]
  {\normalfont\Huge\bfseries\singlespacing}{\thechapter.}{1em}{\Huge}
\titlespacing*{\chapter}{0pt}{-62pt}{0pt}

\titleformat{\section}[block]
  {\normalfont\Large\bfseries}{\thesection.}{4pt}{\Large}
\titlespacing*{\section}{0pt}{\baselineskip}{0pt}

\titleformat{\subsection}[block]
  {\normalfont\large\bfseries}{\thesubsection.}{4pt}{\normalsize\large}
\titlespacing*{\subsection}{0pt}{0pt}{0pt}

\titleformat{\subsubsection}[block]
  {\normalfont\normalsize\bfseries}{\thesubsubsection.}{4pt}{\normalsize}
\titlespacing*{\subsubsection}{0pt}{0pt}{0pt}

\def\tablename{Tabla}

%% Variables para portada y cabeceras
%% Cambiar los valores para cada documento!!!
\def\title{TÍTULO DEL PROYECTO}
\def\subject{ASIGNATURA/SUBTÍTULO}
\def\author{Juan Francisco Mier Montoto}
\def\target{www.mier.info}
\def\authorid{UO283319}
\def\date{mes AÑO}
\def\org{Escuela Politécnica de Ingeniería de Gijón}
\def\area{Grado en Ingeniería Informática en Tecnologías de la Información}

\def\ORG{\expandafter\MakeUppercase\expandafter{\org}}
\def\AREA{\expandafter\MakeUppercase\expandafter{\area}}
\def\SUBJECT{\expandafter\MakeUppercase\expandafter{\subject}}

\captionsetup{justification=centering}
\setlength{\headheight}{65pt}

\fancyhf{}
\fancyhead[L]{\includegraphics[height=16mm]{style/square.png}
  \hspace{1em} \Longstack[l] {
    \textbf{\SUBJECT} \newline
    \textbf{\title}}
  \newline \leftmark{}
}
\fancyhead[R]{\bfseries{Hoja \hyperlink{toc}{\thepage}~de~\pageref{LastPage}}}
\fancyfoot[C]{\href{\target}{\author}}
\renewcommand{\headrulewidth}{0pt} % default is 0pt
\renewcommand{\footrulewidth}{0.4pt} % default is 0

\fancypagestyle{plain}{%
  \fancyhf{}
  \fancyhead[L]{\includegraphics[height=16mm]{style/square.png}
    \hspace{1em} \Longstack[l]{
      \textbf{\SUBJECT} \newline
      \textbf{\title}}}
  \fancyhead[R]{\bfseries{Hoja \hyperlink{toc}{\thepage}~de~\pageref{LastPage}}}
  \fancyfoot[C]{\href{\target}{\author}}
  \renewcommand{\headrulewidth}{0pt} % default is 0pt
  \renewcommand{\footrulewidth}{0.4pt} % default is 0pt
}

\pagestyle{fancy}

\restylefloat{table}

% comandos de referencia cruzada
\newcommand*{\fullref}[1]{\textit{\hyperref[{#1}]{\ref*{#1} \nameref*{#1}}}}
\newcommand*{\halfref}[2]{\texttt{\hyperref[{#1}]{#2}}}

% página en blanco sin número
\newcommand{\blankpage}{
  \newpage
  \thispagestyle{empty}
  \mbox{}
  \newpage
}

% estilo de código por defecto
\lstdefinestyle{default}{
  basicstyle=\ttfamily\footnotesize,
  breakatwhitespace=false,
  breaklines=true,
  captionpos=b,
  keepspaces=true,
  numbers=left,
  numbersep=5pt,
  numberstyle=\tiny\color{black},
  showspaces=false,
  showstringspaces=false,
  showtabs=false,
  tabsize=2,
  backgroundcolor=\color{backcolour},
  postbreak=\mbox{\textcolor{red}{$\hookrightarrow$}\space},
}

\lstset{style=default}

% estilo de código para yaml
\lstdefinestyle{yaml}{
  basicstyle=\ttfamily\footnotesize\color{darkgreen}\bfseries,
  rulecolor=\color{black},
  string=[s]{'}{'},
  stringstyle=\color{darkgreen},
  comment=[l]{:},
  commentstyle=\color{blue},
  morecomment=[l]{-},
  numbers=left,
  numbersep=5pt,
  numberstyle=\tiny\color{black},
  backgroundcolor=\color{backcolour},
  breaklines=true,
  captionpos=b,
  keepspaces=true,
  showspaces=false,
  showstringspaces=false,
  showtabs=false,
  tabsize=2,
  breakatwhitespace=true,
  postbreak=\mbox{\textcolor{red}{$\hookrightarrow$}\space},
}
