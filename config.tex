\pgfplotsset{compat=1.18}
\usepgfplotslibrary{dateplot}

% colores
\definecolor{gris}{RGB}{220,220,220}
\definecolor{backcolour}{rgb}{0.95,0.95,0.92}
\definecolor{darkgreen}{rgb}{0.0, 0.5, 0.0}
\definecolor{codepurple}{rgb}{0.58, 0.0, 0.83}
\definecolor{codeblue}{rgb}{0.0, 0.0, 0.8}
\definecolor{codegray}{rgb}{0.5, 0.5, 0.5}
\definecolor{codeorange}{rgb}{0.8, 0.4, 0.0}

\setcounter{secnumdepth}{3} % Para permitir numerar las sub-subsecciones

% Modifica el nombre de los índices al castellano
\addto\captionsspanish{
  \renewcommand{\contentsname}{Índice de contenido}
  \renewcommand{\listfigurename}{Índice de figuras}
  \renewcommand{\listtablename}{Índice de tablas}
  \renewcommand{\lstlistingname}{Listado}
  \renewcommand{\lstlistlistingname}{Índice de listados}
}

% Formateo de los nombres de los apartados:
\titleformat{\chapter}[block]
  {\normalfont\Huge\bfseries\singlespacing}{\thechapter.}{1em}{\Huge}
\titlespacing*{\chapter}{0pt}{-62pt}{0pt}

\titleformat{\section}[block]
  {\needspace{12\baselineskip}\normalfont\Large\bfseries}{\thesection.}{4pt}{\Large}
\titlespacing*{\section}{0pt}{\baselineskip}{0pt}

\titleformat{\subsection}[block]
  {\needspace{14\baselineskip}\normalfont\large\bfseries}{\thesubsection.}{4pt}{\normalsize\large}
\titlespacing*{\subsection}{0pt}{0pt}{0pt}

\titleformat{\subsubsection}[block]
  {\needspace{8\baselineskip}\normalfont\normalsize\bfseries}{\thesubsubsection.}{4pt}{\normalsize}
\titlespacing*{\subsubsection}{0pt}{0pt}{0pt}

\def\tablename{Tabla}

%% Variables para portada y cabeceras
%% Cambiar los valores para cada documento!!!
\def\title{TÍTULO DEL PROYECTO}
\def\subject{ASIGNATURA/SUBTÍTULO}
\def\author{Juan Francisco Mier Montoto}
\def\target{www.mier.info}
\def\authorid{UO283319}
\def\date{mes AÑO}
\def\org{Escuela Politécnica de Ingeniería de Gijón}
\def\area{Grado en Ingeniería Informática en Tecnologías de la Información}

\def\ORG{\expandafter\MakeUppercase\expandafter{\org}}
\def\AREA{\expandafter\MakeUppercase\expandafter{\area}}
\def\SUBJECT{\expandafter\MakeUppercase\expandafter{\subject}}

\captionsetup{justification=centering}
\setlength{\headheight}{65pt}

\fancyhf{}
\fancyhead[L]{\includegraphics[height=16mm]{style/square.png}
  \hspace{1em} \Longstack[l] {
    \textbf{\SUBJECT} \newline
    \textbf{\title}}
  \newline \leftmark{}
}
\fancyhead[R]{\bfseries{Hoja \hyperlink{toc}{\thepage}~de~\pageref{LastPage}}}
\fancyfoot[C]{\href{\target}{\author}}
\renewcommand{\headrulewidth}{0pt} % default is 0pt
\renewcommand{\footrulewidth}{0.4pt} % default is 0

\fancypagestyle{plain}{%
  \fancyhf{}
  \fancyhead[L]{\includegraphics[height=16mm]{style/square.png}
    \hspace{1em} \Longstack[l]{
      \textbf{\SUBJECT} \newline
      \textbf{\title}}}
  \fancyhead[R]{\bfseries{Hoja \hyperlink{toc}{\thepage}~de~\pageref{LastPage}}}
  \fancyfoot[C]{\href{\target}{\author}}
  \renewcommand{\headrulewidth}{0pt} % default is 0pt
  \renewcommand{\footrulewidth}{0.4pt} % default is 0pt
}

\pagestyle{fancy}

\restylefloat{table}

% comandos de referencia cruzada
\newcommand*{\fullref}[1]{\textit{\hyperref[{#1}]{\ref*{#1} \nameref*{#1}}}}
\newcommand*{\halfref}[2]{\texttt{\hyperref[{#1}]{#2}}}

% página en blanco sin número
\newcommand{\blankpage}{
  \newpage
  \thispagestyle{empty}
  \mbox{}
  \newpage
}

% estilo de código por defecto
\lstdefinestyle{default}{
  basicstyle=\ttfamily\footnotesize,
  keywordstyle=\color{codeblue}\bfseries,
  stringstyle=\color{codepurple},
  commentstyle=\color{codegray}\itshape,
  numberstyle=\tiny\color{codegray},
  breakatwhitespace=false,
  breaklines=true,
  captionpos=b,
  keepspaces=true,
  numbers=left,
  numbersep=5pt,
  showspaces=false,
  showstringspaces=false,
  showtabs=false,
  tabsize=2,
  backgroundcolor=\color{backcolour},
  rulecolor=\color{gris},
  frame=single,
  xleftmargin=4pt,
  xrightmargin=4pt,
  aboveskip=8pt,
  belowskip=8pt,
  postbreak=\mbox{\textcolor{red}{$\hookrightarrow$}\space},
}

\lstset{style=default}

%% definiciones de lenguajes no soportados nativamente por listings

\lstdefinelanguage{JavaScript}{
  morekeywords={break, case, catch, class, const, continue, debugger, default,
    delete, do, else, enum, export, extends, finally, for, function, if,
    implements, import, in, instanceof, interface, let, new, of, package,
    private, protected, public, return, static, super, switch, this, throw,
    try, typeof, var, void, while, with, yield, async, await, from, as},
  morestring=[b]",
  morestring=[b]',
  morestring=[b]`,
  morecomment=[l]{//},
  morecomment=[s]{/*}{*/},
  sensitive=true,
}

\lstdefinelanguage{TypeScript}[]{JavaScript}{
  morekeywords={type, namespace, declare, module, abstract, readonly,
    keyof, infer, unique, unknown, never, any, boolean, number, string,
    symbol, bigint, object, undefined, null, asserts, is, override,
    satisfies, using},
}

\lstdefinelanguage{Go}{
  morekeywords={break, case, chan, const, continue, default, defer, else,
    fallthrough, for, func, go, goto, if, import, interface, map, package,
    range, return, select, struct, switch, type, var, nil, true, false,
    iota, append, cap, close, copy, delete, len, make, new, panic, print,
    println, recover, error, string, int, int8, int16, int32, int64,
    uint, uint8, uint16, uint32, uint64, float32, float64, complex64,
    complex128, byte, rune, bool, any},
  morestring=[b]",
  morestring=[b]`,
  morecomment=[l]{//},
  morecomment=[s]{/*}{*/},
  sensitive=true,
}

\lstdefinelanguage{Rust}{
  morekeywords={as, break, const, continue, crate, else, enum, extern,
    false, fn, for, if, impl, in, let, loop, match, mod, move, mut,
    pub, ref, return, self, Self, static, struct, super, trait, true,
    type, unsafe, use, where, while, async, await, dyn, abstract,
    become, box, do, final, macro, override, priv, typeof, unsized,
    virtual, yield, i8, i16, i32, i64, i128, isize, u8, u16, u32, u64,
    u128, usize, f32, f64, bool, char, str, String, Vec, Option, Result,
    Some, None, Ok, Err},
  morestring=[b]",
  morecomment=[l]{//},
  morecomment=[s]{/*}{*/},
  sensitive=true,
}

\lstdefinelanguage{Kotlin}{
  morekeywords={abstract, actual, annotation, as, break, by, catch, class,
    companion, const, constructor, continue, crossinline, data, delegate,
    do, dynamic, else, enum, expect, external, false, field, final,
    finally, for, fun, get, if, import, in, infix, init, inline, inner,
    interface, internal, is, it, lateinit, noinline, null, object, open,
    operator, out, override, package, private, protected, public, reified,
    return, sealed, set, super, suspend, tailrec, this, throw, true, try,
    typealias, typeof, val, var, vararg, when, where, while},
  morestring=[b]",
  morestring=[b]',
  morecomment=[l]{//},
  morecomment=[s]{/*}{*/},
  sensitive=true,
}

\lstdefinelanguage{Swift}{
  morekeywords={actor, any, associatedtype, async, await, break, case,
    catch, class, continue, default, defer, deinit, do, else, enum,
    extension, fallthrough, false, fileprivate, for, func, guard, if,
    import, in, init, inout, internal, is, let, nil, open, operator,
    optional, override, private, protocol, public, repeat, required,
    rethrows, return, self, Self, some, static, struct, subscript, super,
    switch, throw, throws, true, try, typealias, var, weak, where, while},
  morestring=[b]",
  morecomment=[l]{//},
  morecomment=[s]{/*}{*/},
  sensitive=true,
}

\lstdefinelanguage{JSON}{
  morestring=[b]",
  literate=
    *{:}{{{\color{codeblue}:}}}{1}
    {,}{{{\color{codeblue},}}}{1}
    {\{}{{{\color{codeblue}\{}}}{1}
    {\}}{{{\color{codeblue}\}}}}{1}
    {[}{{{\color{codeblue}[}}}{1}
    {]}{{{\color{codeblue}]}}}{1}
    {true}{{{\color{codeorange}true}}}{4}
    {false}{{{\color{codeorange}false}}}{5}
    {null}{{{\color{codeorange}null}}}{4},
}

\lstdefinelanguage{Dockerfile}{
  morekeywords={FROM, AS, RUN, CMD, LABEL, MAINTAINER, EXPOSE, ENV, ADD,
    COPY, ENTRYPOINT, VOLUME, USER, WORKDIR, ARG, ONBUILD, STOPSIGNAL,
    HEALTHCHECK, SHELL},
  morestring=[b]",
  morestring=[b]',
  morecomment=[l]{\#},
  sensitive=false,
}

\lstdefinelanguage{HCL}{
  morekeywords={resource, data, variable, output, locals, module, provider,
    terraform, backend, required_providers, required_version, count,
    for_each, depends_on, lifecycle, provisioner, connection, dynamic,
    content, source, version, type, default, description, sensitive,
    validation, nullable, value, true, false, null},
  morestring=[b]",
  morestring=[b]',
  morecomment=[l]{\#},
  morecomment=[l]{//},
  morecomment=[s]{/*}{*/},
  sensitive=true,
}

\lstdefinelanguage{TOML}{
  morestring=[b]",
  morestring=[b]',
  morecomment=[l]{\#},
  literate=
    *{true}{{{\color{codeorange}true}}}{4}
    {false}{{{\color{codeorange}false}}}{5},
  sensitive=true,
}

% estilo de código para yaml (hereda de default)
\lstdefinestyle{yaml}{
  style=default,
  basicstyle=\ttfamily\footnotesize\color{darkgreen}\bfseries,
  string=[s]{'}{'},
  stringstyle=\color{darkgreen},
  comment=[l]{:},
  commentstyle=\color{codeblue},
  morecomment=[l]{-},
  breakatwhitespace=true,
}
